\documentclass[10pt,landscape,a4paper,oneside]{article}

%dbfgdfgdf
%dbfgdfgdf
%dbfgdfgdf
%dbfgdfgdf
%dbfgdfgdf
%dbfgdfgdf
%lol, windows newline bug poje to vrsto na začetku včasih :P
%\documentclass[10pt,a4paper,oneside]{book}

%\usepackage[slovene]{babel}
\usepackage[utf8x]{inputenc}
%\usepackage[fleqn]{amsmath}%fleqn - levo zamaknjene enačbe
\usepackage{amsfonts}
\usepackage{amssymb}
\usepackage{makeidx}
\usepackage{graphicx}
%\usepackage[x11names, rgb]{xcolor}
%\usepackage{tikz}
%\usetikzlibrary{calc,automata,snakes,arrows,shapes,decorations,positioning,shapes.arrows,chains,shadows}

\newenvironment{items}{
\begin{itemize}
	\setlength{\itemsep}{2pt}
	\setlength{\parskip}{0pt}
	\setlength{\parsep}{0pt}
	\setlength{\topsep}{0pt}
}{\end{itemize}}


\begin{document}
\begin{titlepage}
\begin{center}
\ \\[1cm]
{\Huge\bf Operacijski sistemi 2}\\[1cm]
{\resizebox{13cm}{!}{\bf{Mouse gestures for Linux}}}\\[2cm]
{\Large\bf Miha Zidar}\\[0.5cm]
{\Large\bf Anže Pečar}\\[0.5cm]
{\Large\bf Matic Potočnik}\\[2cm]

{\huge \today}\ \\[1.55cm]

\end{center}
\end{titlepage}
\pagebreak
\begin{center}
	\Huge\mbox{\bf Opis zamisli}\\[2cm]
	\begin{items}
	\item Program teče v ozadju in spremlja gibanje miške
	\item Začetek geste označuje posebna kombinacija tipk npr. super+right-click
	\item Če program prepozna storjeno gesto, izvede določeno akcijo
	\item Preko grafičnega vmesnika je možnost ustvarjanja gest in dodajanja akcij
	\end{items}
\end{center}
\pagebreak
\begin{center}
	\Huge\mbox{\bf Osnovni cilji}\\[2cm]
	\begin{items}
	\item \mbox{Olajašati rokovanje z računalnikom}
	\item \mbox{Pohitritev pogosto rabljenih akcij}
	\item \mbox{Narediti uporabno aplikacijo}
	\end{items}
\end{center}
\begin{center}
	\Huge\mbox{\bf Osnovne zahteve in omejitve}\\[2cm]
	\begin{items}
	\item \mbox{Preprečiti je potrebno privzeto akcijo}
	\item \mbox{Možnost nastavljanja različnih gest in profilov}
	\end{items}
\end{center}


\end{document}
