\documentclass[10pt,a4paper,oneside]{book}

\usepackage[slovene]{babel}
\usepackage[utf8x]{inputenc}
\usepackage{makeidx}
\usepackage{graphicx}
\usepackage{nag}
\usepackage[margin=1in]{geometry}
\usepackage{enumerate}
\usepackage{verbatim}

\usepackage{listingsutf8}
\lstset{
	tabsize=4,
	basicstyle={\ttfamily \scriptsize},
	identifierstyle={\ttfamily},
	inputencoding=cp1250,
	%barvanje kode
	commentstyle={\sffamily \color[rgb]{0,0.5,0}},
	stringstyle=\color[rgb]{0.5,0,1},
	keywordstyle=\color[rgb]{0,0,1},
	classoffset=0,
	%barvanje kode
	language=Java,
	deletekeywords={true,false},
	deletekeywords={byte,short,int,long,float,double,char,boolean,void},
	classoffset=1,
	morekeywords={byte,short,int,long,float,double,char,boolean,void},
	keywordstyle=\color[rgb]{0.5,0,1},
	%iz source datotek potegnemo tisto med //@begin@ ...koda... //@end@
	rangeprefix=//@,
	rangesuffix=@,
	linerange=begin-end,
	includerangemarker=false,
	%lomljenje predolgih vrstic
	breaklines=true,
	prebreak=\raisebox{0ex}[0ex][0ex]{\ensuremath{\hookleftarrow}},
	%okvir
	frame=single,
	frameround={t}{t}{t}{t},
	xleftmargin=23pt,
	xrightmargin=4pt,
	framexleftmargin=19pt,
	framerule=0pt,
	%številčenje
	numbers=left,
	firstnumber=1
}

\usepackage[svgnames]{xcolor}
\usepackage{fancybox}
\usepackage{varwidth}
\newcommand\inline[1]{%
\begin{Sbox}{#1}\end{Sbox}%
\colorbox{lightgray}{\TheSbox}%
}
\newcommand\inliney[1]{%
\begin{Sbox}\begin{varwidth}{0.91\textwidth}{#1}\end{varwidth}\end{Sbox}%
\colorbox{lightgray}{\TheSbox}%
}

\newcommand\pic[2]{%
\parbox{1cm}{%
\begin{center}%
\includegraphics[width=#1\textwidth]{#2}%
\end{center}%
}%
}

\newcommand\br{%? any better ideas? check \par
 \ \\ \ \\%
}

\usepackage{hyperref}
\hypersetup{
    colorlinks=true,
    linkcolor=black,
    urlcolor=blue,
    pdftitle={gms - Generic Mouse Gestures}
}
\title{gms - Generic Mouse Gestures}

\begin{document}
\begin{titlepage}
\begin{center}
\ \\[3cm]
{\resizebox{15cm}{!}{\Huge gms}}\\[-40pt]
{\huge\bf \ \ \ \ \ \ \ \ \ \ \ Generic Mouse Gestures}\\[4.5cm]
{\Huge\bf Dokumentacija}\ \\[0.35cm]
{\huge \today}\ \\[3cm]
{\huge {\bf Avtorji:}}\\[0.35cm]
{\Large Anže Pečar, Miha Zidar, Matic Potočnik}
\vfill
\end{center}
\end{titlepage}
\tableofcontents
\pagebreak

\chapter{Uvod}
\section{O programu}
\inline{gms} je program za upravljanje računalnika s pomočjo miškinih gest. Ustvarjen je bil v sklopu seminarske naloge za predmet \inline{Operacijski Sistemi 2} na univerzitetnem programu Fakultete za računalništvo in informatiko v Ljubljani. Repozitorij z izvorno kodo je na voljo na: \inline{\url{https://github.com/Smotko/gms}}
\chapter{Navodila za uporabo}
\section{Namestitev}
Prikazana je namestitev na sistem \inline{Ubuntu 11.04 Natty Narwhal}.
\subsection{Odvisnosti}
Program ima nekaj odvisnosti, ki niso pokrite v privzeti namestitvi Ubuntu 11.04.\\
To popravimo z namestitvijo paketov \inline{python-xlib} \inline{lineakd}, ter \inline{git}:\\
\inliney{sudo apt-get install python-xlib lineakd git}
\subsection{Namestitev}
Kot je navada v Ubuntuju, program naložimo v mapo \inline{/usr/local/bin}:\\
\inliney{cd /usr/local/bin\\
sudo git clone git://github.com/Smotko/gms.git}
\br\br
Če program želimo zaganjati iz kjerkoli, moramo njegovo mapo dodati v iskalno pot.\\To naredimo tako, da dodamo naslednji vrstici v datoteko \inline{$\sim$/.bash\_rc}:\\
\inliney{PATH=\$PATH:/usr/local/bin/gms\\
export PATH}

\subsection{Posodabljanje}
Če želimo program kdaj posodobiti, v terminalu zaženemo:\\
\inliney{cd /usr/local/bin/gms\\
sudo git pull}
\section{Uporaba}
\subsection{Zagon}
Za zagon preprosto zaženemo \inline{gms} v terminalu, po želji pa si lahko ustvarimo tudi bližnjico v kakšnem izmed menijev, ali pa program dodamo med programe, ki se zaženejo ob zagonu operacijskega sistema.

\subsection{Uporaba}
Ko program zaženemo, se v zgornji vrstici v kotičku za obveščanju pojavi ikona.\\
\pic{0.9}{./gmsAppindicator.png}\\
S klikom na ikono se odpre meni, s katerim lahko pridemo do nastavitev, podatkov o programu, ali pa zapremo program.
\br
Program nato uporabljamo prek miškinih gest.\\
Podprtih je naslednjih 8 gest:\\
\pic{0.2}{./gmsGestures.png}
\begin{itemize}
\item Gor--Levo
\item Gor
\item Gor--Desno
\item Desno
\item Dol-Desno
\item Dol
\item Dol--Levo
\item Levo
\end{itemize}
Akcije gest so nastavljive (glej \ref{Nastavitve}), obsegajo pa povsem poljubne pritiske tipk tipkovnice. Tako lahko upravljamo z vsakim programom, ki podpira bližnjice na tipkovnici, nekatere bližnjice (npr. za zapiranje programov), pa so podprte že s strani operacijskega sistema.
\newpage
\subsection{Nastavitve}\label{Nastavitve}
Do nastavitev pridemo s klikom na ikono v zgornji vrstici in izbiro \inline{Preferences}\\
\pic{0.9}{./gmsAppindicator.png}\\
Pojavi se okno:\\
\pic{0.5}{./gmsSettings.png}
\br
S klikom na gumb ob \inline{Set default key} nastavimo, katero tipko je potrebno držati med izvajanjem geste, pri čemer privzeta nastavitev \inline{False} pomeni, nobene.
\br
S \inline{Set mouse button} nastavimo, katero miškino tipko je potrebno držati med izvajanjem geste.
\br
S klikom na \inline{Config gestures} pa se odpre okno, kjer nastavimo katera akcija se bo izvedla ob kateri gesti.\br
\pic{0.5}{./gmsKeys.png}\\

\chapter{Tehnična zasnova}
\section{Uporabljene tehnologije}
Program gms je napisan v programskem jeziku \inline{Python}. Za uporabniški vmesnik nastavitev je uporabljeno ogrodje \inline{GTK}, za dodajanje ikone v kotiček za obveščanje je uporabiljen \inline{IndicatorApplet}, za zajemanje premikov miške in tipk tipkovnice in tudi za izvajanje akcij, pa smo uporabili knjižnico \inline{Xlib}. Za prepoznavanje miškinih gest smo uporabili lasten algoritem.\\[1.6cm]
\pic{0.5}{./logoPython.png}\ \ \ \ \ \ \ \ \ \ \ \ \ \ \ \ \ \ \ \ \ \ \ \ \ \ \ \ \ \ \ \ \ \ \ \ \ \ \ \ \ \ \ \ \ \ \ \ \ \ \ \ \ \ \ \ \ \ \ \ \ \ \ \ \ \ \ \ \ \ \ \ \ \ \ \ \ \ \ \ \ \ \ \ \ \ \ \ \ 
\pic{0.2}{./logoGTK.jpg}\\[1cm]
\. \ \ \ \ \ \ \ \ \ \ \ \ \ \ \ \ \ \ \ \ \ \ \ \pic{0.25}{./logoX.png}\ \ \ \ \ \ \ \ \ \ \ \ \ \ \ \ \ \ \ \ \ \ \ \ \ \ \ \ \ \ \ \ \ \ \ \ \ \ \ \ \ \ \ \ \ \ \ \ \ \ \ \ 
\pic{0.15}{./logoAppindicator.png}


\end{document}